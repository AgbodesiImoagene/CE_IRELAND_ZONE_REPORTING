\section{Non-Functional Requirements}

This section defines the non-functional requirements (NFRs) that govern the
quality attributes of the Christ Embassy Ireland Zone Church Reporting Platform.
These requirements ensure that the system operates securely, reliably, and
efficiently across all sub-domains and user roles.

\subsection{Security}

\begin{itemize}
    \item \textbf{NFR-S1} — All user authentication and session management shall
    follow industry-standard security practices using HTTPS, salted password
    hashing (Argon2 or bcrypt), and optional Single Sign-On (SSO) for pastors.
    \item \textbf{NFR-S2} — Role-Based Access Control (RBAC) and organisational
    scopes shall be enforced both at the API layer and the database layer via
    Row-Level Security (RLS).
    \item \textbf{NFR-S3} — All data in transit shall be encrypted using TLS
    1.2 or higher; data at rest shall be encrypted using AES-256 or equivalent.
    \item \textbf{NFR-S4} — Multi-factor authentication (2FA) shall be required
    for pastors, finance officers, and users with elevated permissions.
    \item \textbf{NFR-S5} — The platform shall maintain immutable audit logs for
    all sensitive actions including data exports, deletions, financial
    verifications, and role assignments.
    \item \textbf{NFR-S6} — Session timeouts shall occur after 30 minutes of
    inactivity or 12 hours of continuous login, whichever comes first.
    \item \textbf{NFR-S7} — Password reset tokens shall expire within 15 minutes
    and be single-use.
\end{itemize}

\subsection{Privacy \& Compliance}

\begin{itemize}
    \item \textbf{NFR-P1} — The system shall comply with the General Data
    Protection Regulation (GDPR) as applicable in Ireland and the EU.
    \item \textbf{NFR-P2} — All personally identifiable information (PII) shall
    be encrypted at rest and masked where not essential to a user’s role.
    \item \textbf{NFR-P3} — Consent for storing contact information and
    follow-up communication shall be explicitly captured during registration.
    \item \textbf{NFR-P4} — Users with \texttt{system.exports.full\_pii} must
    complete step-up 2FA confirmation before export.
    \item \textbf{NFR-P5} — Members shall have the right to request data
    correction or deletion, subject to church policy and record-keeping
    obligations.
    \item \textbf{NFR-P6} — Data retention periods shall be configurable, with a
    default of 7 years for audit and financial records.
\end{itemize}

\subsection{Reliability \& Availability}

\begin{itemize}
    \item \textbf{NFR-R1} — The system shall achieve a minimum of 99.9\%
    uptime, excluding scheduled maintenance.
    \item \textbf{NFR-R2} — The platform shall degrade gracefully in the event
    of partial component failure (e.g., temporary loss of reporting service
    shall not affect data entry in Registry or Finance portals).
    \item \textbf{NFR-R3} — All user operations shall be idempotent where
    applicable (e.g., duplicate uploads or repeated submissions shall not
    produce duplicate records).
    \item \textbf{NFR-R4} — The system shall perform daily backups of the
    database and retain rolling backups for a minimum of 30 days.
    \item \textbf{NFR-R5} — Backups shall be encrypted and stored in geographically
    redundant locations.
    \item \textbf{NFR-R6} — Recovery Point Objective (RPO): 1 hour; Recovery Time
    Objective (RTO): 2 hours.
    \item \textbf{NFR-R7} — A backup restore test shall be successfully executed
    at least once per quarter.
\end{itemize}

\subsection{Performance \& Scalability}

\begin{itemize}
    \item \textbf{NFR-PS1} — API response times shall not exceed 300 ms for
    95\% of requests under normal load (P95 latency).
    \item \textbf{NFR-PS2} — The system shall support at least 1,000 concurrent
    active users within a single zone tenant.
    \item \textbf{NFR-PS3} — Data queries and reports shall use indexed
    aggregations to return results within 2 seconds for standard filters.
    \item \textbf{NFR-PS4} — The system shall scale horizontally via container
    replication (Docker/Kubernetes) and vertically via resource provisioning.
    \item \textbf{NFR-PS5} — Background jobs such as imports, exports, and
    scheduled report generation shall be handled asynchronously via a job queue
    (e.g., Celery or RQ) to avoid blocking requests.
    \item \textbf{NFR-PS6} — File uploads and large exports shall use presigned
    URLs or streamed downloads to optimise throughput.
\end{itemize}

\subsection{Maintainability \& Extensibility}

\begin{itemize}
    \item \textbf{NFR-M1} — The codebase shall follow a modular architecture,
    separating backend services, shared components, and frontend portals.
    \item \textbf{NFR-M2} — The backend shall expose a documented REST API with
    OpenAPI/Swagger specification.
    \item \textbf{NFR-M3} — All configuration values (e.g., database URIs, email
    credentials) shall be injected via environment variables or secure
    parameter storage.
    \item \textbf{NFR-M4} — New portals or modules (e.g., Events, Outreach) can
    be added without modifying existing schemas or permissions, using the same
    tenant-based architecture.
    \item \textbf{NFR-M5} — Automated testing (unit, integration, and API-level)
    shall be maintained with a minimum 80\% coverage target.
    \item \textbf{NFR-M6} — Continuous Integration/Deployment (CI/CD) shall be
    implemented with code linting, schema migrations, and automated rollback.
    \item \textbf{NFR-M7} — All dependencies shall be pinned to versions and
    checked for security vulnerabilities regularly.
\end{itemize}

\subsection{Usability \& Accessibility}

\begin{itemize}
    \item \textbf{NFR-U1} — All portals shall be responsive and usable on
    desktop, tablet, and mobile screens.
    \item \textbf{NFR-U2} — Interfaces shall use clear typography and layout with
    consistent navigation across portals.
    \item \textbf{NFR-U3} — Data-entry forms shall include validation feedback
    and contextual error messages.
    \item \textbf{NFR-U4} — Colour contrast and text size shall meet WCAG 2.1 AA
    accessibility guidelines.
    \item \textbf{NFR-U5} — The system shall provide user feedback for all
    asynchronous operations (e.g., import progress, batch locking).
\end{itemize}

\subsection{Monitoring \& Observability}

\begin{itemize}
    \item \textbf{NFR-O1} — Application logs shall be structured (JSON) and
    centralised for analysis.
    \item \textbf{NFR-O2} — Metrics such as API latency, job queue depth, and
    failed logins shall be captured and visualised on dashboards.
    \item \textbf{NFR-O3} — Alerts shall be configured for availability,
    performance degradation, and authentication anomalies.
    \item \textbf{NFR-O4} — Audit and monitoring data shall be retained for at
    least 12 months.
\end{itemize}
