\section{User Journeys and Workflows}

This section outlines the key user workflows within the Christ Embassy Ireland
Zone Church Reporting Platform. Each workflow describes how a typical task is
performed from start to finish, highlighting role interactions, system
validations, and audit behaviour.

\subsection{Overview}

Workflows are grouped by portal and role. Each represents a real-world ministry
process captured within the system and emphasises data integrity, accountability,
and reporting traceability.

\subsection{Registry Workflows}

\paragraph{1. First-Timer Registration and Assimilation}

\begin{enumerate}
    \item A Church Administrator logs in to the Registry Portal and selects the
    relevant church.
    \item After a service, the clerk records first-timers:
    \begin{itemize}
        \item Full name, gender, phone/email, inviter/source.
        \item Service attended (linked via \texttt{service\_id}).
        \item Initial status: \texttt{New}.
    \end{itemize}
    \item The first-timer record appears in the “Follow-Up Pipeline” dashboard.
    \item When follow-up occurs, a pastoral staff member updates status to
    \texttt{Contacted} or \texttt{Returned}.
    \item If the person joins a foundation class, the system allows conversion
    to a Member record:
    \begin{itemize}
        \item The user clicks “Convert to Member.”
        \item A new \texttt{people} record is created, copying contact info and
        linking to the original first-timer entry.
    \end{itemize}
    \item The workflow ends when status becomes \texttt{Member} and is reflected
    in assimilation KPIs.
\end{enumerate}

\paragraph{2. Service Attendance Recording}

\begin{enumerate}
    \item Church Administrator selects a \texttt{service\_date} and \texttt{service\_name}
    (Sunday, Midweek, or Special).
    \item Attendance counts are entered for men, women, teens, and children, as
    well as totals for first-timers and new converts.
    \item The system validates one attendance record per service per church.
    \item On submission, an attendance summary is written to the
    \texttt{attendance} table.
    \item Background job updates
    \texttt{summary\_attendance\_daily(org\_unit\_id, date)}.
    \item Zonal and Group Pastors immediately see updated totals in their
    dashboards.
\end{enumerate}

\paragraph{3. Member Record Update}

\begin{enumerate}
    \item Clerks or pastors edit member details (address, marital status,
    foundation school completion, baptism date, etc.).
    \item All edits are logged to \texttt{audit\_logs} with before/after snapshots.
    \item Updates trigger recalculation of membership demographics and KPIs.
\end{enumerate}

\subsection{Finance Workflows}

\paragraph{1. Offering Entry and Batch Locking}

\begin{enumerate}
    \item Finance Officer creates a new batch for a church and service.
    \item Within the batch, each entry records:
    \begin{itemize}
        \item Transaction date, amount, fund, partnership arm, method, member or
        cell reference, and comment.
    \end{itemize}
    \item Entries remain in \texttt{draft} status until verified.
    \item When verification is complete:
    \begin{itemize}
        \item The first authorised user with
        \texttt{finance.batches.lock} attempts to lock the batch.
        \item The system requests a second approver (dual control).
        \item Upon dual confirmation, the batch status updates to
        \texttt{locked}.
    \end{itemize}
    \item Once locked:
    \begin{itemize}
        \item Entries become immutable.
        \item Dashboards and summaries refresh automatically.
        \item Audit trail logs both approvers’ actions.
    \end{itemize}
\end{enumerate}

\paragraph{2. Partnership Tracking}

\begin{enumerate}
    \item Pastor or Finance Officer creates partnership records for members,
    specifying:
    \texttt{fund\_id}, \texttt{partnership\_arm\_id}, pledge amount, and cadence.
    \item Finance entries linked to the same member and partnership arm
    increment fulfilment totals automatically.
    \item The partnership dashboard displays current fulfilment percentage,
    outstanding balance, and due dates.
\end{enumerate}

\paragraph{3. Reconciliation Workflow}

\begin{enumerate}
    \item Zonal or Group Finance Team exports a verification summary by fund and
    church.
    \item After confirming bank deposits, the team marks verified batches as
    \texttt{reconciled}.
    \item Locked and reconciled batches contribute to financial KPIs.
\end{enumerate}

\subsection{Cells Workflows}

\paragraph{1. Weekly Cell Report Submission}

\begin{enumerate}
    \item Cell Leader logs in to the Cells Portal and selects their assigned
    cell.
    \item Fills in:
    \begin{itemize}
        \item Date and meeting time.
        \item Attendance, first-timers, new converts.
        \item Testimonies and notes.
        \item Offerings total.
        \item Meeting type (prayer, Bible study, outreach).
    \end{itemize}
    \item On submission:
    \begin{itemize}
        \item Report status = \texttt{submitted}.
        \item Offerings total triggers an automated creation of a
        \texttt{finance\_entry} with
        \texttt{source\_type='cell\_report'} and
        \texttt{source\_id=cell\_report.id}.
        \item The finance entry appears in Finance Portal under the respective
        church for verification.
    \end{itemize}
    \item Pastors can review and approve reports (optional workflow step),
    changing status to \texttt{approved}.
\end{enumerate}

\paragraph{2. Cell Reporting Compliance}

\begin{enumerate}
    \item System checks each active cell weekly.
    \item Cells without reports are marked “missing submission.”
    \item Reports Portal dashboard visualises submission compliance by church
    and group.
\end{enumerate}

\subsection{Reports Portal Workflows}

\paragraph{1. Dashboard Interaction and Drill-Down}

\begin{enumerate}
    \item User logs in and the system determines accessible scope via assigned
    roles.
    \item Dashboard displays summary KPIs:
    attendance, giving, membership, cell compliance.
    \item User drills down:
    \texttt{Zone → Group → Church → Cell → Member}.
    \item Filters: date range, fund, service, or partnership arm.
    \item Data sourced from materialized summary tables.
\end{enumerate}

\paragraph{2. Report Export}

\begin{enumerate}
    \item User applies filters and clicks “Export.”
    \item If small ($\leq$10,000 rows), export is synchronous; otherwise, a background
    job is queued.
    \item Export file (CSV, Excel, or PDF) generated and stored in S3.
    \item User notified with download link; audit log entry created.
\end{enumerate}

\paragraph{3. Scheduled Reporting}

\begin{enumerate}
    \item Pastor configures recurring reports (e.g., Weekly or Monthly Summary).
    \item Job scheduler triggers export and sends email via background worker.
    \item The \texttt{exports} table logs delivery time, recipient, and success
    status.
\end{enumerate}

\subsection{Administrative Workflows}

\paragraph{1. User and Role Provisioning}

\begin{enumerate}
    \item Zonal Pastor creates Group Pastors and assigns group scopes.
    \item Group Pastors create Church Pastors/Coordinators.
    \item Church Pastors create Church Administrators, Finance Officers, and Cell
    Leaders.
    \item All actions audited with user, scope, and timestamp.
\end{enumerate}

\paragraph{2. Importing Legacy Data}

\begin{enumerate}
    \item Admin uploads legacy spreadsheet (e.g., membership list).
    \item System displays field mapping preview.
    \item Validation engine checks for missing or invalid data.
    \item User corrects and re-uploads if errors exist.
    \item Upon successful import:
    \begin{itemize}
        \item Records inserted under correct church/org scope.
        \item Audit log created with import job reference.
    \end{itemize}
\end{enumerate}

\subsection{Workflow Consistency Rules}

\begin{itemize}
    \item Every workflow action must be traceable in \texttt{audit\_logs}.
    \item Data transitions follow strict states:
    \begin{itemize}
        \item \textbf{Finance:} draft → verified → reconciled → locked
        \item \textbf{Cells:} submitted → reviewed → approved
        \item \textbf{First-timers:} new → contacted → returned → member
    \end{itemize}
    \item State changes are immutable in reverse unless unlocked by authorised
    personnel with dual control.
\end{itemize}
