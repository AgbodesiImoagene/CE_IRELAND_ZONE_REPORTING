\section{Functional Requirements}

This section defines the functional requirements for the Christ Embassy Ireland Zone
Church Reporting Platform. Requirements are grouped by portal (Registry,
Finance, Cells, Reports) and describe system behaviour, data relationships, and
user interactions. Each requirement is uniquely identified for reference.

\subsection{Registry Portal (\texttt{registry.<zone>.ce.church})}

\paragraph{Purpose}
Provides authorised church administrators with facilities to manage member
records, first-timers, service attendance, and departmental assignments.

\paragraph{Core Entities}
\begin{itemize}
    \item \textbf{Member (Person):} uniquely identified individual with
    demographic and contact details, membership status, and church affiliation.
    \item \textbf{First Timer:} a visitor who attended for the first time and may
    later be linked to a Member record.
    \item \textbf{Service Attendance:} a record of attendance counts for a
    specific service date and church.
    \item \textbf{Department:} organisational unit within a church with members
    and leaders.
\end{itemize}

\paragraph{Requirements}
\begin{itemize}
    \item \textbf{FR-R1} — The system shall allow Church Administrators and authorised
    pastors to create, edit, and search Member records containing:
    name, aliases, gender, date of birth, contact information, address, town,
    county, Eircode, marital status, foundation school completion, baptism date,
    membership status, and associated cell.
    \item \textbf{FR-R2} — The system shall validate mandatory fields (first
    name, last name, gender, contact number or email, and assigned church).
    \item \textbf{FR-R3} — Each Member record shall have a unique internal
    identifier and audit trail of changes.
    \item \textbf{FR-R4} — Church Administrators shall be able to register first-timers
    and associate them with the service they attended, capturing name, contact,
    inviter/source, and notes.
    \item \textbf{FR-R5} — The system shall maintain a pipeline of first-timers
    with statuses: New → Contacted → Returned → Member.
    \item \textbf{FR-R6} — The system shall store attendance summaries for each
    service, including men, women, teens, kids, total attendance, first timers,
    new converts, and notes. Each attendance entry is linked to a Service object
    (\texttt{service\_id}) identifying the date, time, and church.
    \item \textbf{FR-R7} — The system shall provide validation to ensure only one
    attendance record per service per church.
    \item \textbf{FR-R8} — Departmental assignments shall link Members to one or
    more departments and optionally mark a leader.
    \item \textbf{FR-R9} — Registry users shall be able to export membership or
    attendance data as CSV/Excel, limited by their scope.
    \item \textbf{FR-R10} — All Registry actions (create, update, delete, merge)
    shall be recorded in the audit log.
\end{itemize}

\subsection{Finance Portal (\texttt{finance.<zone>.ce.church})}

\paragraph{Purpose}
Captures and verifies records of all forms of monetary giving for accounting and
reporting purposes. The system does \emph{not} process payments but maintains a
structured record of financial transactions and partnerships.

\paragraph{Core Entities}
\begin{itemize}
    \item \textbf{Fund:} category of giving (tithe, offering, seed, first fruit,
    partnership, etc.).
    \item \textbf{Partnership Arm:} specific ministry initiative such as
    Rhapsody of Realities, The Healing School, InnerCity Mission for Children,
    Loveworld Television Ministry, or any custom campaign.
    \item \textbf{Finance Entry:} individual record of a contribution linked to
    a member, cell, or external giver.
    \item \textbf{Batch:} collection of finance entries for a service; may be
    locked and verified.
\end{itemize}

\paragraph{Requirements}
\begin{itemize}
    \item \textbf{FR-F1} — Finance Officers shall be able to record new finance
    entries with: transaction date, amount, fund, partnership arm, payment
    method (cash, KingsPay, bank transfer, cheque, POS, other), reference,
    member/cell linkage, and comment.
    \item \textbf{FR-F2} — Each finance entry shall reference the associated
    service where applicable (\texttt{service\_id}).
    \item \textbf{FR-F3} — The system shall permit association of entries to
    either a Member (\texttt{person\_id}), a Cell (\texttt{cell\_id}), or an
    external entity (text name).
    \item \textbf{FR-F4} — Entries shall include a \texttt{verified\_status}
    field with states: Draft → Verified → Reconciled → Locked.
    \item \textbf{FR-F5} — Each batch shall be uniquely identified by
    \texttt{(church, service\_date, service\_name)} and can be locked only after
    dual approval.
    \item \textbf{FR-F6} — Locked batches are immutable; unlock requires dual
    authorisation and is fully audited.
    \item \textbf{FR-F7} — The system shall allow recording of partnerships with
    amount, start date, frequency (weekly, monthly, quarterly, annual), and
    fulfilment percentage.
    \item \textbf{FR-F8} — Finance users shall be able to generate summaries per
    fund, partnership arm, church, or service.
    \item \textbf{FR-F9} — All finance data exports must include generation
    timestamp, scope, and user metadata.
    \item \textbf{FR-F10} — Only authorised users can edit or delete verified
    entries, and all such actions require justification text.
\end{itemize}

\subsection{Cells Portal (\texttt{cells.<zone>.ce.church})}

\paragraph{Purpose}
Allows Cell Leaders to submit weekly cell reports and track their own cell
activities, while allowing pastors to monitor participation and outreach.

\paragraph{Core Entities}
\begin{itemize}
    \item \textbf{Cell:} a small fellowship group linked to a church with leader,
    assistant, meeting day, time, and venue.
    \item \textbf{Cell Report:} a weekly or ad-hoc meeting report containing
    attendance counts, first timers, new converts, testimonies, offerings, and
    meeting type.
\end{itemize}

\paragraph{Requirements}
\begin{itemize}
    \item \textbf{FR-C1} — Cell Leaders shall be able to submit cell reports for
    their assigned cell only.
    \item \textbf{FR-C2} — Each cell report shall include date, time, meeting
    type, attendance, first timers, new converts, testimonies, and offerings
    total.
    \item \textbf{FR-C3} — Each cell report shall automatically reference its
    parent cell (\texttt{cell\_id}) and the church through that cell’s
    \texttt{org\_unit\_id}.
    \item \textbf{FR-C4} — The system shall validate that only one report per
    cell per date can be submitted.
    \item \textbf{FR-C5} — Cell offerings shall be automatically included in the
    Finance reports under the corresponding fund when applicable.
    \item \textbf{FR-C6} — Pastors may view submission status and overdue
    reports across cells they oversee.
    \item \textbf{FR-C7} — The system shall support optional approval workflow
    for cell reports (submitted → reviewed → approved).
\end{itemize}

\subsection{Reports Portal (\texttt{reports.<zone>.ce.church})}

\paragraph{Purpose}
Provides zonal, group, and local pastors with dashboards and drill-down reports
spanning membership, finance, attendance, and cell activities.

\paragraph{Requirements}
\begin{itemize}
    \item \textbf{FR-P1} — The system shall display KPIs and charts based on the
    user’s scope:
    \begin{itemize}
        \item Attendance trends (weekly, monthly, demographic splits)
        \item First-timer assimilation funnel
        \item Foundation school and baptism progress
        \item Department participation rate
        \item Giving trends by fund and partnership arm
        \item Per-capita giving (total giving / average attendance)
        \item Cell reporting compliance and outreach outcomes
    \end{itemize}
    \item \textbf{FR-P2} — Dashboards shall provide drill-down from zone → group
    → church → cell → individual.
    \item \textbf{FR-P3} — Reports shall support filters by date range, fund,
    church, cell, and member status.
    \item \textbf{FR-P4} — Reports shall be exportable as CSV, Excel, or PDF.
    \item \textbf{FR-P5} — Users with appropriate permissions may schedule
    recurring email reports.
\end{itemize}

\subsection{Data Import \& Migration}

\paragraph{FR-I1} The system shall support import of legacy spreadsheets
(Members, Services, Attendance, Finance, Cells, Cell Reports) through a guided
mapping interface with preview and validation.

\paragraph{FR-I2} Validation errors shall be reported with line numbers and
descriptions. Users may download a corrected template for re-upload.

\paragraph{FR-I3} Imports shall be performed per portal according to role
permissions; only Zonal or Group Pastors can perform zone-wide imports.

\subsection{Audit \& Logging Integration}

\paragraph{FR-A1} All create, update, delete, merge, lock/unlock, verify, and
export actions shall be written to an immutable audit log with timestamp, actor,
entity type, entity id, and IP address.

\paragraph{FR-A2} System shall expose audit entries to users with
\texttt{system.audit.view} permission filtered by their scope.

\paragraph{FR-A3} Audit logs shall be retained for a minimum of seven years and
included in backups.
